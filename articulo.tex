

\documentclass[twoside,twocolumn]{article}

\usepackage{blindtext} % Package to generate dummy text throughout this template 
\usepackage{graphicx}
\usepackage[sc]{mathpazo} % Use the Palatino font
\usepackage[T1]{fontenc} % Use 8-bit encoding that has 256 glyphs
\linespread{1.05} % Line spacing - Palatino needs more space between lines
\usepackage{microtype} % Slightly tweak font spacing for aesthetics

\usepackage[english]{babel} % Language hyphenation and typographical rules

\usepackage[hmarginratio=1:1,top=32mm,columnsep=20pt]{geometry} % Document margins
\usepackage[hang, small,labelfont=bf,up,textfont=it,up]{caption} % Custom captions under/above floats in tables or figures
\usepackage{booktabs} % Horizontal rules in tables

\usepackage{lettrine} % The lettrine is the first enlarged letter at the beginning of the text

\usepackage{enumitem} % Customized lists
\setlist[itemize]{noitemsep} % Make itemize lists more compact

\usepackage{abstract} % Allows abstract customization
\renewcommand{\abstractnamefont}{\normalfont\bfseries} % Set the "Abstract" text to bold
\renewcommand{\abstracttextfont}{\normalfont\small\itshape} % Set the abstract itself to small italic text

\usepackage{titlesec} % Allows customization of titles
\renewcommand\thesection{\Roman{section}} % Roman numerals for the sections
\renewcommand\thesubsection{\roman{subsection}} % roman numerals for subsections
\titleformat{\section}[block]{\large\scshape\centering}{\thesection.}{1em}{} % Change the look of the section titles
\titleformat{\subsection}[block]{\large}{\thesubsection.}{1em}{} % Change the look of the section titles

\usepackage{fancyhdr} % Headers and footers
\pagestyle{fancy} % All pages have headers and footers
\fancyhead{} % Blank out the default header
\fancyfoot{} % Blank out the default footer
\fancyhead[C]{Sistema para la gestion de Configuración de Software $\bullet$ Octubre 2019 $\bullet$ } % Custom header text
\fancyfoot[RO,LE]{\thepage} % Custom footer text

\usepackage{titling} % Customizing the title section

\usepackage{hyperref} % For hyperlinks in the PDF

%----------------------------------------------------------------------------------------
%	TITLE SECTION
%----------------------------------------------------------------------------------------

\setlength{\droptitle}{-4\baselineskip} % Move the title up

\pretitle{\begin{center}\Huge\bfseries} % Article title formatting
\posttitle{\end{center}} % Article title closing formatting
\title{Sistema para la Gestión de Configuración de Software} % Article title
\author{Sigfredo Aponte, Alonso Andia, Diego Porlles}
\date{\today} % Leave empty to omit a date
\renewcommand{\maketitlehookd}{%
\begin{abstract}
\noindent 
El presente documento tiene por función proveer una visión general de la arquitectura del Sistema para la Gestión de Configuración de Software (SGCS), plasmando la idea general del proyecto, objetivos y problematica.
\end{abstract}
\begin{abstract}
\noindent 
The purpose of this document is to provide an overview of the architecture of the System for Software Configuration Management (SGCS), reflecting the general idea of ​​the project, objectives and problems.
\end{abstract}
}
%----------------------------------------------------------------------------------------
\begin{document}
% Print the title
\maketitle
%----------------------------------------------------------------------------------------
%	ARTICLE CONTENTS
%----------------------------------------------------------------------------------------
\section{Titulo del Proyecto}
\lettrine[nindent=0em,lines=3]{S}istema para la Gestion de Configuración de Software.



%-------------------------------Marco Teórico-----------------

\section{Planteamiento del Problema}
\subsection{Problema}
\subsubsection{El Problema}
Actualmente no se cuenta con un sistema que pueda ser capaz de ver las versiones y los nuevos cambios lo cual causa deficiencia en los diferentes procesos.

\subsubsection{¿A quien Afecta?}
A los miembros de alguna empresa que se dedique a los proyecto de software y también a los clientes.

\subsubsection{¿Cual es su Impacto?}

	Demora en el proceso de revisar los cambios y nuevas versiones realizados por los miembros del proyecto. 
\subsubsection{Solución Planteada}
	Automatizar el proceso de gestión de proyectos que sea capaz de llevar un control apropiado de los mismos y actualice los resultados de avance. 


\subsection{Justificación}
La Justificación principal de este sistema es automatizar y optimizar los procesos de gestión de usuario, gestión de proyecto, generación de las solicitudes de cambio y gestionar las solicitudes de revisión.  De esta manera se pueda llevar un control más rápido de las diferentes actividades de cada miembro de un proyecto.

\subsection{Alcance}
El sistema SGCS (Sistema de Gestión de Configuración de Software) que se desarrollará se encargará de identificar los ECS y de los cambios que se necesitaran realizar en los proyectos de desarrollo de SW, además de visualizar los informes de estado además de los avances por fase de cada metodología empleada.

%--------------------------------Fin Marco Teórico----------------

%-----------------------------------------------------------------
\section{Objetivos}\label{sec:5}

\subsection{General}
El Sistema Gestión de Cambios de Software (SGCS) tiene como objetivo principal automatizar la Gestión de la configuración de un proyecto de software automatizando y reduciendo tiempos, de esta manera agilizar los procesos de los proyectos.
\subsection{Especifico}
\begin{itemize}	
	\item Crear un software que permita la gestión de documentos y las versiones de éste.
	\item Crear un software que permita la gestión de Usuarios.
	\item Crear un software que permita mostrar los proyectos y ver su estado.
	\item Crear un software que permita la generación de solicitud de revisión.
	\item Crear un software que permita la generación de solicitud de cambio.
\end{itemize} 
%-----------------------------------------------------------------
\section{Referentes Teoricos}
\begin{itemize}
	\item Libro1. Diseño e implementación del proceso de gestión de la configuración de software en la empresa de desarrollo venture venti.
	\item Libro2. Sistema de gestión de la configuración del software y de despliegue para la plataforma PIO
	\item Libro3. Sistema Evolutivo de Gestión de la configuración del software.
	\item Libro4. Técnicas Cuantitativas para la Gestión en la Ingeniería del Software.
	\item Libro5. Gestión de la configuración
	\item Libro6. Sistema para gestión de configuración de software
	
	
\end{itemize}
\section{Desarrollo de la Propuesta}
\subsection{Tecnologia de información }
\begin{itemize}
	\item Base de Datos SQL Server.
	\item Lenguaje de Programación en PHP.
	\item Arquitectura Modelo Vista Controlador
	\item HTML, CSS, JavaScript.	
	
	
\end{itemize}
\subsection{Metodologia, Tecnicas Usadas }
El grupo para este proyecto vio como mas conveniente utilizar la metodología RUP para la realización del proyecto.
\section{Cronograma}
\begin{itemize}
\item Se adjunta el cronograma en la carpeta del documento por motivos prácticos.


\end{itemize}




\end{document}
